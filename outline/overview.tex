\documentclass{article}

\usepackage[left=2.5cm,top=3cm,right=2.5cm,bottom=3cm,bindingoffset=0.5cm]{geometry}
\usepackage{amsmath}
\usepackage{amssymb}
\usepackage{physics}
\usepackage{color}
\usepackage{graphicx}
\usepackage{../macros/macros_custom}

\author{Kraig Andrews}
\title{Prospectus Outline}
\date{\today}

\begin{document}
\maketitle

\begin{enumerate}%outer layer begin

	\item{\textbf{Introduction}}
	\begin{enumerate}%begin
		\item{\emph{The Conception of Semiconductors}}
		\begin{enumerate}%begin
			\item{Basics of semiconductors date founded on the pivotal foundations laid by 18th and 19th century physicists (e.g. Faraday and Volta) \cite{Busch_EuroJournPhys1989,Faraday_Royal1833, Volta_Royal1782}.}
			\item{The term semiconductor, in the sense that it is known today, appeared in a paper by Koenigsberger in 1911 \cite{Koenigsberger_AnnalenDerPhysik1911}.}
			\item{Much of the scientific community remained pessimistic towards the viability and usefulness of semiconductors up until the years following WWII \cite{Busch_EuroJournPhys1989}.}
			\item{Some of the theoretical framework, presenting the ideas of \emph{instrinsic} and \emph{extrinsic} semiconductors, were presented by Wilson (takes into account donors and acceptors) \cite{Wilson_Royal1931a, Wilson_Royal1931b}.}
			\item{Post-WWII research on germanium and silicon began to shed light on the possibilities of semiconductors, for example the work by Lark-Horovitz shed light on the existence of \emph{intrinsic} semiconductors \cite{Lark_AAAS1954, Busch_EuroJournPhys1989}.}
		\end{enumerate}%end
		
		\item{\emph{Evolution of Semiconductors}}
		\begin{enumerate}%begin
			\item{First transistor is constructed at Bell Labs in 1947 by Shockley, Bardeen, and Brattian (device was polycrystalline germanium, shortly thereafter it was also developed using silicon) \cite{Neamen_Semiconductor_Physics2003}.}
			\item{These first devices were eventually improved on by implementing single crystal materials instead of polycrystalline. This greatly improved the properties of the semiconductor device \cite{Neamen_Semiconductor_Physics2003}.}
			\item{Use of diffusion process to form junctions. Allows for better control and allows for higher-frequency devices. Allows many transistors to be made on a single silicon slice, reducing cost of devices. Become commercially available in late 1950s \cite{Neamen_Semiconductor_Physics2003}.}
			\item{In 1958 the first integrated circuit (IC) was demonstrated (Jack Kilby of TI), he would eventually receive the Nobel Prize in physics for this \cite{Lukasiak_JorunTelcomm2010, Kilby_Patent1959}. }
			\item{In the years that followed significant improvements were made. The scale of ICs grew rapidly (Moore's law) \cite{Moore_Electronics1965}.}
			\item{From being able to fit a few transistors on a chip in the beginning of the 1960s (small-scale integration) to present-day chips billions of transistors on one chip \cite{Clarke_EEtimes2005}.}
			\item{As growth of the semiconductor industry continued, limits on device integration and material limits of silicon and other commonly used materials are looming, causing interest in search for new materials \cite{Meindl_Science2001, Schulz_Nature1999}. }
			\item{These limitations, in part, drove the increased interest in alternative materials to traditional semiconductor materials. As a result of this, widespread research has been conducted on several new materials that are renewed interest. }
		\end{enumerate}%end
		
		\item{\emph{Development of \Td Materials}}
		\begin{enumerate}%begin
			\item{}
		\end{enumerate}%end

		\item{\emph{Current State of \Td Materials}}
		\begin{enumerate}%begin
			\item{}
		\end{enumerate}%end
	\end{enumerate}%end

	\item{\textbf{Experimental Details}}

	\item{\textbf{Preliminary Results and Discussion}}

	\item{\textbf{Future Works}}

\end{enumerate} %outer layer end

%------- Bib(s) ------------
\bibliographystyle{plain}
\bibliography{../refs/database}

\end{document}
