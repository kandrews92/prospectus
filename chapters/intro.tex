\chapter{Introduction}\label{sec:intro}
\section{Birth of Integrated Circuits}
The semiconductor revolutionized the world in the latter half of the twentieth century. The term semiconductor, in the sense it is known today, first appears in literature in 1911 \cite{Koenigsberger_AnnalenDerPhysik1911}. Initially, work on the subject was rather pessimistic. However, in the years following Word War II breakthroughs began shed light on the possible applications and the underlying physics involved, such as the ideas of \emph{instrinsic} and \emph{extrinsic} semiconductors \cite{Busch_EuroJournPhys1989,Lark_AAAS1954,Wilson_Royal1931a,Wilson_Royal1931b}. 

The history of semiconductors and transistors is a well documented subject. The first transistor was constructed at Bell Labs in 1947 using polycrystalline germanium. Shortly thereafter one was developed using silicon. Throughout the following years, these devices were improved on by replacing polycrystalline with single crystals \cite{Neamen_Semiconductor_Physics2003}. Then Jack Kilby demonstrated the first integrated circuit (IC) in 1958, for which he would win the Nobel Prize in physics \cite{Lukasiak_JorunTelcomm2010, Kilby_Patent1959}. The scale of ICs grew rapidly in the subsequent years. Initially only a few transistors could fit on a chip (small-scale integration), in stark contrast to moder-day chips that contains billions of transistors \cite{Moore_Electronics1965, Clarke_EEtimes2005}. Growth continued at a rapid pace, but eventually it was realized that some limits, material and integration based, existed in silicon and other commonly used materials \cite{Meindl_Science2001, Schulz_Nature1999}. In part, these limitations increased the interest in alternative materials. As a result widespread and renewed interest has led to a breadth information and results on a wide range of materials and their applications.

\section{Graphene as a New Two-Dimensional Material}\label{sec:graphene}