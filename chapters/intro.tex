\chapter{Introduction}\label{sec:intro}
\section{Birth of Integrated Circuits}
The development of microelectronics revolutionized the world in the latter half of the twentieth century. The term semiconductor, in the sense it is known today, first appears in literature in 1911 \cite{Koenigsberger_AnnalenDerPhysik1911}. Initially, work on the subject was rather pessimistic. However, in the years following Word War II breakthroughs began shed light on the possible applications and the underlying physics involved, such as the ideas of \emph{instrinsic} and \emph{extrinsic} semiconductors \cite{Busch_EuroJournPhys1989,Lark_AAAS1954,Wilson_Royal1931a,Wilson_Royal1931b}. \\

\noindent The history of semiconductors and transistors is a well documented subject. The first transistor was constructed at Bell Labs in 1947 using polycrystalline germanium. Shortly thereafter one was developed using silicon. Throughout the following years, these devices were improved on by replacing polycrystalline with single crystals \cite{Neamen_Semiconductor_Physics2003}. Then Jack Kilby demonstrated the first integrated circuit (IC) in 1958, for which he would win the Nobel Prize in physics \cite{Lukasiak_JorunTelcomm2010, Kilby_Patent1959}. The scale of ICs grew rapidly in the subsequent years. Initially only a few transistors could fit on a chip (small-scale integration), in stark contrast to moder-day chips that contains billions of transistors \cite{Moore_Electronics1965, Clarke_EEtimes2005}. Growth continued at a rapid pace, but eventually it was realized that some limits, material and integration based, existed in silicon and other commonly used materials \cite{Meindl_Science2001, Schulz_Nature1999}. In part, these limitations increased the interest in alternative materials. As a result widespread and renewed interest has led to a breadth information and results on a wide range of materials and their applications.

\section{Graphene as a New \Td Material}\label{sec:graphene}
Layered materials have existed for billions of years, and have been studied over the last few centuries \cite{Golden_EarthSci2013,Brodie_Royal1859}. In recent decades the scientific study of graphite (3D) has led to new forms materials, such as carbon nanotubes (1D) and fullerenes (0D) \cite{Kroto_Nature1985, Balleste_Nanoscale2011,Iijima_Nature1991}. However, only more recently have scientists began to understand the potential of such layered materials and their potential technological applications. After attempting unsuccessfully to synthesize few-layer graphite during the 1960s (only around 10-50 layers were able to be synthesized) a breakthrough was finally acheived \cite{Balleste_Nanoscale2011}. This most notably began with the synthesis of monolayer graphene \cite{Novoselov_Science2004}.

\subsection{Properties of Graphene}\label{subsec:graphene_properties}
To date, graphene's properties have been the focus of much research, both theoretical and experimental. It has been one of the primary driving forces in study of `relativistic' condensed matter physics due to its low dimensionality and its band structure that allows electrons to mimic relativistic particles confirming the appearance of several relativistic phonomena \cite{Geim_NatureMat2007,Geim_Nature2005,Zhang_NatPhys2011,Williams_Science2007}. In its most basic sense, graphene is composed of a single layer of carbon atoms arranged in \td honeycomb lattice (see fig.~\ref{fig:graphene_honeycomb}) It has a Young's modulus of $~100\unita{GPa}$ (several times more than steel) with a breaking force that is $13\%$ of its Young's modulus \cite{Bertolazzi_ACSnano2011, Akinwande_NatureComm2014}. Its stength is due, in part, to its strong in-plane carbon (\ch{C}) bonds. In addition, graphene can sustain elastic deformations of 20\% due to its \td nature and it has high pliability \cite{Balleste_Nanoscale2011}. These mechanical properties are of interest because graphene lies in the extreme ranges of many metrics considering its size and dimensionality.\\

\noindent Aside from its mechanical properties, graphene's transport properties were another reason why the material was so appealing. Graphene's mobility is several times that of silicon's. Experimental results have shown graphene mobility around $15,000\cmvs$ with a potential theoretical limit of $200,000\cmvs$ \cite{Dargys_Encylco1994,Akinwande_NatureComm2014}. The upper theoretical limit imposed on mobility is due to scattering, however, these high mobilities are achieved mainly because electrons in graphene act very much like photons in their mobility due to their lack of mass. This enables them to travel sub-micron distances without scattering \cite{Novoselov_NatureMat2007}. In reality, there are other limiting factors that need to be considered such as the quality of graphene and scattering with the substrate, for example. \\

\begin{figure}[ht]
	\centering
	\begin{minipage}[b]{0.45\linewidth}
		\includegraphics[height=4cm,width=5cm]{figs/graphene_honeycomb}
		\caption[Graphene honeycomb lattice]{Graphene: a layer of carbon atoms in a honeycomb lattice.}
		\label{fig:graphene_honeycomb}
	\end{minipage}
%\end{figure}
%\begin{figure}[ht]
	\qquad
	\begin{minipage}[b]{0.45\linewidth}
		\includegraphics[height=4cm,width=6cm]{figs/graphene_bandgap}
		\caption[Bandgap of graphene]{One of the most unusual features of graphene is that its conduction and valence bands meet at a point, meaning that in single-layer graphene there is no band gap (figure obtained from \cite{Berkley_Online2009})}
		\label{fig:graphene_bandgap}
	\end{minipage}
\end{figure}

\noindent Despite its impressive properties, the main drawback of graphene is its lack of bandgap. As this became known, the prospect of using graphene for the fabrication to ICs became unlikely. In graphene the conduction and valence bands touch at a single point as shown in fig.~\ref{fig:graphene_bandgap} \cite{Wallace_PhysRev1947}. Ultimately, the lack of a bandgap means that the current on/off ratio is low and is unappealing for logical circuit applications \cite{Xu_ChemRev2013}. However, graphene exhibits some interesting properties as a result of having no bandgap, particularly as it pertains to its optical properties. The material's band structure allows for absorption of light over a large range of the electromagnetic spectrum, ranging from infrared ($<1.65\unita{eV}$) to ultraviolet ($>3.2\unita{eV}$), offering potential electronic-photonic device applications \cite{Xia_NatureNano2009,Wang_Science2008,Geim_NatureComm2011}.
