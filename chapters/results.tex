\chapter{Studying Instrinsic Channel Properties, Scattering Mechanisms, and Quantum Transport}\label{chap:results}
Developing low resistance two-dimensional/two-dimensional (2D/2D) ohmic contacts opens up possibility to study the intrinsic properties of \acp{TMD} and quantum physics. In particular, quantum phenomena inherent to \acp{2DEG} and \acp{2DHG} such as the integers and fractional quantum Hall effects and \ac{SdH} oscillations can be explored in high mobility monolayer and few-layer \acp{TMD} \cite{Cui_NatureNano2015}. In addition to quantum transport properties and quantum effects in monolayer and few-layer \acp{TMD} the study of mobility and its corresponding temperature dependence can be used to understand the multiple scattering mechanisms present \cite{Kaasbjerg_PhysRevB2012}. These study of both electron and hole transport mechanisms is important due to the fact that high-performance $p$-type and $n$-type transistors are necessary for complimentary digital applications. 

\section{$p$-type \ch{MoS2} Semiconductor Doping}\label{sec:mos2_doping}
One of the major challenges that still remains in fabricating devices to study quantum transport and scattering mechanisms is developing high quality $p$-type \ch{MoS2} devices. This is due to the fact that the metal/\ch{MoS2} interface is obstructed by a large \ac{SB} formed by Fermi level pinning close to the conduction band of the \ch{MoS2} \cite{Chuang_NanoLett2014,Das_NanoLett2012}.
