\chapter{Experimental Details}\label{chap:exp_details}

\section{Substrate Preperation}\label{sec:sample_prep}
Using degenerately doped \ac{SiO2} wafers that are $270\unita{nm}$ thick as pictured in fig.~\ref{fig:plain_wafer} and the subsequent substrate`s schematic in fig.~\ref{fig:si_sio2_diagram}, there are several preliminary steps needed prior to device fabrication. For easy indentification of locations on the substrate alignment marks are placed on the wafer using photolithography (see setup in fig.~\ref{fig:photolithography_bay}). There is a main alignment mark pictured in figs.~\subref*{fig:main_alignment_5x} and \subref*{fig:main_alignment_10x} which allows for quicker indentification during electron beam lithography, for example. The alignment marks are in a grid pattern with the coordinate $\left(0,0\right)$ at the center, stretching to $\left(\pm 6,\pm 6\right)$ in both the right and left directions. In each of these coordinate locations there smaller alignment marks evenly spaced within them as shown in figs.~\subref*{fig:main_alignment_50x} and \subref*{fig:main_alignment_100x}. Next, \ac{Au} is deposited on the surface of the wafer, a process that will be explained in more detail in sec.~\ref{sec:device_fabrication}.
\begin{figure}[ht]
	\centering
	\begin{minipage}[b]{0.25\linewidth}
		\centering
		\includegraphics[height=2.5cm,width=2.5cm]{figs/experimental/plain_wafer}
		\caption[Plain wafer]{Plain, polished uncut \ch{Si}/\ch{SiO2} wafer.}
		\label{fig:plain_wafer}
	\end{minipage}
	\qquad
	\begin{minipage}[b]{0.25\linewidth}
		\centering
		\includegraphics[height=1.75cm,width=3.25cm]{figs/experimental/si_sio2_diagram}
		\caption[Schematic of \ch{Si}/\ch{SiO2} substrate]{Schmatic of \ch{Si}/\ch{SiO2} substrate.}
		\label{fig:si_sio2_diagram}
	\end{minipage}
	\qquad
	\begin{minipage}[b]{0.25\linewidth}
		\centering
		\includegraphics[height=2cm,width=3cm]{figs/experimental/photolithography_bay}
		\caption[Photolithography system]{Photolithography system for creating alignment marks on substrates.}
		\label{fig:photolithography_bay}
	\end{minipage}
\end{figure}
~

\begin{figure}[ht]
	\centering
	\subfloat[]{
		\includegraphics[height=4cm,width=5cm]{figs/experimental/main_alignment_5x}
		\label{fig:main_alignment_5x}
	}
	\qquad
	\subfloat[]{
		\includegraphics[height=4cm,width=5cm]{figs/experimental/main_alignment_10x}
		\label{fig:main_alignment_10x}
	}

	\subfloat[]{
		\includegraphics[height=4cm,width=5cm]{figs/experimental/main_alignment_50x}
		\label{fig:main_alignment_50x}
	}
	\qquad
	\subfloat[]{
		\includegraphics[height=4cm,width=5cm]{figs/experimental/main_alignment_100x}
		\label{fig:main_alignment_100x}
	}
	\caption[Alignment marks at varying magnifications]{\protect\subref{fig:main_alignment_5x} Main alignment mark at 5x magnification. \protect\subref{fig:main_alignment_10x} Main alignment mark at 10x magnification. \protect\subref{fig:main_alignment_50x} Coordinate mark $\left(0,-5\right)$ at 50x magnification. \protect\subref{fig:main_alignment_100x} Coordinate mark $\left(0,-5\right)$ at 100x magnification.}
	\label{fig:main_alignment}
\end{figure}

\subsection{Substrate Cleaning}\label{subsec:cleaning}
First, a \ch{SiO2} wafer is cut to the appropriate size with its preexisting \acs{Au} layer. To remove the \acs{Au} layer, the substrate is first soaked in acetone for approximately 5-10 minutes then washed using \ac{IPA} and dried with \ac{N2} gas. Next, the substrate in placed in acetone and sonicated for 15 minutes. Then sonicated once more but in \acs{IPA} this time with a repetition of washing and drying step using \ac{IPA} and \acs{N2} as described above in between each sonication. In order to remove any remaining organic matter on the surface of the substrate, the substrate is annealed under vacuum at $600^\degree\unita{C}$ for 10 minutes and passing forming gas for 2 of the 10 minutes. Forming gas is a mixture of \ch{H2} and an inert gas, usually \ch{N2} \cite{Choi_AppPhysLett2004}. In addition to annealing the substrate for cleanliness, in certain cases when a higher degree of cleanliness is desired the substrate can be treated with oxygen plasma cleaning. 

\section{Exfoliation}\label{sec:exfoliation}
To synthesize samples the most common and often most effective method used is mechanical exfoliation, a technique made famous by Novoselov \emph{et al.} \cite{Novoselov_Science2004}. The process involves using Scotch tape to repeatedly cleave layers of \acs{MoS2} or some other TMD. Initially a crystal of a particular TMD (see fig.~\subref*{fig:exfoliation_step1}) is placed on a piece of Scotch tape (see fig.~\subref*{fig:exfoliation_step2}). Then taking another piece of tape and pressing in on the crystal that is on the first piece of tape, being sure to press hard and firm on the crystal. The tape is then lifted up and this process is repeated until the whole piece of tape is filled with small samples of the TMD (see fig.~\subref*{fig:exfoliation_step3}). At the end of this process it is expected that there are a wide range of mixture of sample sizes in terms of area and in term of thickness as well, where thicknesses of $<3\unita{nm}$ are not uncommon. To better characterize the samples the optical microscope is used.
\\ \\
\noindent The main challenge that exists with this method is the ability to synthesize a high yield of monolayer samples. This does not seem to be much of a challenge when it comes to graphene and some other TMDs, but with regard to \acs{MoS2} this is not the case. Based on recently published literature in an effort to increase the yield of monolayer \acs{MoS2} various methods and techniques were tested and modified accordingly \cite{Huang_et_al_ACSnano2015}. In this modified method an additional step to cleaning the substrate is added in which it undergoes oxygen plasma cleaning for 10 minutes to ensure the cleanliness of the substrate`s surface. To promote more bonding between the substrate and the samples, the substrate is first heated at $300^\degree\unita{C}$ for 10 minutes without any samples on it. During this process the normal cleaving of sample on tape from crystal takes place. Once the substrate is done heating the tape containing the samples is immediately placed on the substrate and pressed firmly for several minutes. Then the substrate (with the tape still on it) is placed on a glass slide (microscope slide) and is heated at around $85^\degree\unita{C}$ for five minutes. Next, the substrate (with tape) is removed from heat and the tape slowly peeled back from the substrate. The result should be a much higher yield of $<3\unita{nm}$ samples of larger surface area, and several trilayer, bilayer, and a few monolayer samples. 
\begin{figure}[ht]
	\centering
	\subfloat[]{
		\includegraphics[height=2cm,width=3cm]{figs/experimental/exfoliation_step1}
		\label{fig:exfoliation_step1}
	}
	\qquad
	\subfloat[]{
		\includegraphics[height=2cm,width=3cm]{figs/experimental/exfoliation_step2}
		\label{fig:exfoliation_step2}
	}
	\qquad
	\subfloat[]{
		\includegraphics[height=2cm,width=3cm]{figs/experimental/exfoliation_step3}
		\label{fig:exfoliation_step3}
	}
	\caption[Exfoliation steps]{\protect\subref{fig:exfoliation_step1} Bulk \ch{MoS2} crystal. \protect\subref{fig:exfoliation_step2} Single \ch{MoS2} crystal on tape. \protect\subref{fig:exfoliation_step3} Tape with exfoliation \ch{MoS2} crystals.}
	\label{fig:exfoliation_steps}
\end{figure}
\\ \\
\noindent Most commonly \acs{SiO2} substrates are the main items that are exfoliated onto. However, depending on the material being synthesized, this may not always be the case. In cases where samples of \hbn or in the event that the thickness of the synthesized sample is not of great importance and can to tolerated up to $20-30\unita{nm}$, \ac{PDMS} is exfoliated onto instead of \acs{SiO2} substrates. The resulting samples are of varying thickness, on average around $20\unita{nm}$. Thin samples (usually trilayer and above) can be made using this method of \acs{PDMS}, however, these samples tend to have small surface area and lack uniformity which poses problems as to their usability. As such, this remains an effective method for obtaining samples in which thickness is not the main concern. Once the samples have been optically characterized, the sample(s) on \acs{PDMS} must be transferred to a \acs{SiO2} substrate.

\section{Device Synthesis}\label{sec:synthesis}
Once the sample or samples have been synthesized and characterized for their specific purpose these samples can begin to be synthesized into a device for measurement. Generally, this involves the technique of transfer. Transfer is usually done using the aforementioned \acs{PDMS} or using \ac{PC} known as the \acs{PC} pickup method. 
%
\subsection{\acs{PDMS} Transfer}\label{subsec:pdms_transfer}
\acs{PDMS} transfer is most useful for samples that were originally exfoliated onto \acs{PDMS}, for example \hbn. To manufacture \acs{PDMS} a 10:1 ratio of silicone base and curing agent is mixed together and placed in vacuum for 30 minutes to ensure the removal of any remaining air bubbles. After this time the mixture is then spin coated on a plain \acs{SiO2} wafer and heated at $80^\degree\unita{C}$ for 30 minutes then allowed to cool for 30 minutes. Once cooled, the surface of the wafer can be cut using a razor into small stamps that can used for exfoliation and for transfer. \\ \\

\noindent Once the samples that are to be transferred are on the \acs{PDMS} stamp, it is placed on a glass slide. The optical microscope is used to locate the sample on the stamp and using a razor to cut small excess pieces from the portions of the stamp where the desired sample is not located. This process is repeated until the size of the cut stamp is now reasonably small. The cut stamp is then placed at the edge of a new glass slide with sample area of the stamp as close to the edge as can be and the other side of the stamp is taped down using Scotch tape. \\ \\

\noindent Next a substrate is placed and secured using glue (usually PMMA) to the stage of the transfer setup. The transfer stage setup is pictured in fig.~\ref{fig:transfer_stage_setup}. It consists of a microscope that has the capability of 10x or 20x magnification and a micro-manipulator. The micro-manipulator is where the glass slide with the \acs{PDMS} stamp is placed. Using the manipulator the substrate on the stage is approached and the position of the stamp is checked and re-checked multiple times using the microscope to ensure correct overlap of the desired portion of the sample(s). Upon reaching the desired position, the glass slide is lowered, but this time there should be a contrast seen which is the overlapping of the glass slide and the substrate. Once the contrast has enveloped the entire sample that was to be transferred then the manipulator can be used to lift up the glass slide. Once the transfer is complete then the substrate should be annealed at $250^\degree\unita{C}$ for 30 minutes in order to remove and residue or orgranic matter that may have remained during the transfer process.
\begin{figure}[ht]
	\centering
	\includegraphics[height=5cm,width=7cm]{figs/experimental/transfer_stage_setup}
	\caption[Transfer stage setup]{Transfer stage setup}
	\label{fig:transfer_stage_setup}
\end{figure}
\subsection{Polycarbonate Pickup Method}\label{subsec:pc_pickup}
The PC pickup method is used for samples that have been exfoliated onto a \acs{SiO2} substrate. Generally these are thinner samples with larger surface area that are not as easily obtained by using the PDMS exfoliation method as described in sec.~\ref{sec:exfoliation}. To manufacture the PC $3.0\unita{g}$ of chloroform and $0.18\unita{g}$ of polycarbonate resin are put on a plate shaker for about 60 minutes or until the polycarbonate resin have dissolved into the solution. \\ \\

\noindent Next, the substrate that has the sample that is going to be transferred is taped using double-sided tape to a glass slide facing up. Then using a syringe the \acs{PC} solution is placed in the substrate and evenly spread across it, being sure to locate the area(s) on the substrate where the sample(s) are located, small pre-cut pieces of PDMS are placed over top of these areas. An outline of the \acs{PDMS} stamps is cut using a razor and any excess \acs{PDMS} is carefully torn away. Once only the \acs{PDMS} strips are remaining on the substrate \ac{DI} is put under the strips in order to create a hydrophobic surface and to ensure that the strip and PC that is trapped underneath it come off the substrate with relative ease. Each strip is placed on its own glass slide and is gently blown with \ch{N2} gas to remove any excess \acs{DI} from the surface. \\ \\

\noindent Moving to the transfer stage setup and following the steps described in sec.~\ref{subsec:pdms_transfer} with regard to using the transfer stage setup. The only difference at this point to using this method as opposed to \acs{PDMS} transfer is in the final step of the transfer. Instead of only lowering until the contrast change is shown between the region that is desired to be transferred, with PC transfer the entire PC must be lowered down. This is because the PC will be heated before being lifted up. Lowering all the way ensures that all the PC will be melted. Once lowered all the way, the heating device, which is connected to the stage, should be turned up to $130^\degree\unita{C}$. Once this temperature is reached, it should be maintained for approximately two minutes to fully melt the PC film. After heating the substrate and lifting up using micro-manipulator the substrate is placed in chloroform and covered for 30-60 minutes. The purpose of this is to remove any residue left over by the PC film or any other items that may have been introduced at any point in the transfer process. In practice the chloroform soaking generally needs to be repeated several times over a few hours in order to ensure the least amount of remaining residue possible. To confirm the reduction of residue and also characterize the transferred samples an AFM is used. 

\section{Characterization}\label{sec:characterization}
There are many ways used in modern academia and industry to characterize samples and devices. Some of these methods include \ac{STM}, \ac{TEM}, \ac{AFM}, and \ac{MFM} \cite{Kittel_IntroSolidState2005}. The primary characterization techniques used in this project are \acs{AFM} and optical characterization.
\subsection{Optical Characterization}\label{subsec:characterization_optical}
The majority of the optical characterization is carried out using the optical microscope as shown in figs.~\subref*{fig:optical_microscope_front_view} and~\subref*{fig:optical_microscope_side_view}. The microscope can magnify 5x, 10x, 20x, 50x, and 100x, in addition, it can show dark field images.
\begin{figure}[ht]
	\centering
	\subfloat[]{
		\includegraphics[height=5cm,width=3.5cm]{figs/experimental/optical_microscope_front_view}
		\label{fig:optical_microscope_front_view}
	}
	\qquad
	\subfloat[]{
		\includegraphics[height=5cm,width=3.5cm]{figs/experimental/optical_microscope_side_view}
		\label{fig:optical_microscope_side_view}
	}
	\caption[Optical microscope]{\protect\subref{fig:optical_microscope_front_view} Optical microscope front view. \protect\subref{fig:optical_microscope_side_view} Optical microscope side view.}
\end{figure}

\subsection{AFM Characterization}\label{subsec:characterization_afm}
In addition to the characterizing samples optically, \acs{AFM} characterization is another important aspect of the device design and fabrication process. \acs{AFM} characterization occurs several times throughout the process, after each transfer of a sample onto another, for example. This occurs for two reasons; to verify the thickness of the sample(s) that have been transferred, and also to verify the cleanliness of the surface of the sample (to ensure that any residue has been removed, especially during the course of PC transfer). Additionally, once the electrodes of the device have been fabricated a final \acs{AFM} characterization is need to determine the width of the device's channel which is needed to calculate various important electrical properties. \\ \\

\noindent Fig.~\subref*{fig:afm_front_view} shows a front view of the \acs{AFM} used to characterize. For these purposes, the \acs{AFM} is operating in ``tapping" mode which is less invasive than ``contact" mode \cite{Kittel_IntroSolidState2005}. In basic terms, an \acs{AFM} works by measuring the force between the tip of a cantilever (see fig.~\subref*{fig:AFM_tip}) and the sample being imaged. 
\begin{figure}[ht]
	\centering
	\subfloat[]{
		\includegraphics[height=4cm,width=4cm]{figs/experimental/AFM_front_view}
		\label{fig:afm_front_view}
	}
	\qquad
	\subfloat[]{
		\includegraphics[height=4cm,width=4cm]{figs/experimental/AFM_tip}
		\label{fig:AFM_tip}
	}
	\caption[AFM setup and AFM cantilever]{\protect\subref{fig:afm_front_view} Front view of \acs{AFM} setup. \protect\subref{fig:AFM_tip} AFM cantilever tip \cite{Ernst-Moritz-Arndt_Online}.}
\end{figure}

\section{Device Fabrication}\label{sec:device_fabrication}
In order to perform electrical measurements of devices one must fabricate devices with electrodes. The devices are fabricated according to a specific process, though it is worth noting that under certain circumstances some steps of the process may be omitted or altered, however, the main idea remains the same regardless of the device type being fabricated. Once all transfer steps and samples have been placed in their correct locations the fabrication process begins. This process has three main steps: device design (sec.~\ref{subsec:device_design}), \ac{EBL} (sec.~\ref{subsec:lithography}), and metal deposition (sec.~\ref{subsec:deposition}).
\subsection{Device Design}\label{subsec:device_design}
The specifics of device electrode design depends largely on the type of measurement that is desired. For example, fig.~\ref{fig:ebeam_developed_100x} demonstrates a Hall bar device pattern. Despite the varying possibilities of patterns, the method for which the designs are generated remains largely the same. In order to generate these patterns a design software, \ac{NPGS}, is used to draw the electrodes which in turn communicates the designed pattern to the \acs{SEM} \cite{NPGS}. \\ \\
\noindent In preperation for \acs{EBL} the substrate is spin-coated with two layers of \ac{PMMA}. The first layer of \acs{PMMA} is 495-A4 is added then the substrate is baked on a hot plate for 5 minutes at $180^\degree\unita{C}$. The first layer is followed by another layer of \acs{PMMA}, 950-A2 (same polymer of different molecular weight). Again, the substrate is spin-coated with this \acs{PMMA} layer and baked at the same temperature for 5 minutes.

\subsection{Electron Beam Lithography}\label{subsec:lithography}
To generate the patterns described in sec.~\ref{subsec:device_design} a \acs{SEM} is used to complete \acs{EBL}. Fig.~\ref{fig:SEM_machine} shows the control panel and electron beam writer of the \acs{SEM}. The \acs{SEM} is optimized by adjusting the beam current to the saturation point. Using the alignment marks mentioned in sec.~\ref{sec:sample_prep} the electron beam is aligned to the correct position on the substrate. The \acs{NPGS} system allows for adjustments in the concentration of the electron beam current at certain areas, for example, whether a line dose or an area dose is necessary. In general, these values are adjusted as needed depending on the device design but are usually around $300\unita{\mu C}\unitb{cm}{-2}$ for area doses and $15\unita{\mu C}\unitb{cm}{-1}$ for line doses. Ultimately, the \acs{EBL} process creates three separate patterns that are joined together. The first pattern is written at $1000\mathrm{x}$ magnification on the \acs{SEM} (see fig.~\subref*{fig:ebeam_developed_100x}). The second and third patterns (written at $300\mathrm{x}$ and $100\mathrm{x}$, respectively, see fig.~\subref*{fig:ebeam_developed_10x}) are connected to the inner $1000\mathrm{x}$ pattern and they connect this inner pattern to the ``electrode pads" where the electrodes will eventually be connected for device measurement. After completing the device pattern, the substrate is developed in a solution of \ac{MIBK} and \ac{MEK} for 70 seconds. The \acs{MIBK} is the main developer while the \acs{MEK} acts to enhance the developing process, both fig.~\subref*{fig:ebeam_developed_10x} and fig.~\subref*{fig:ebeam_developed_100x} have undergone this development process.
\begin{figure}[ht]
	\centering
	\includegraphics[height=4cm,width=7cm]{figs/experimental/SEM}
	\caption[Scanning electron microscope]{Control panel and electron beam writer of scanning electron microscope.}
	\label{fig:SEM_machine}
\end{figure}

\begin{figure}[ht]
	\centering
	\subfloat[]{
		\includegraphics[height=4cm,width=5cm]{figs/experimental/ebeam_developed_10x}
		\label{fig:ebeam_developed_10x}
	}
	\qquad
	\subfloat[]{
		\includegraphics[height=4cm,width=5cm]{figs/experimental/ebeam_developed_100x}
		\label{fig:ebeam_developed_100x}
	}
	\caption[Electron beam lithography patterns]{\protect\subref{fig:ebeam_developed_10x} Developed pattern at 10x magnification, \protect\subref{fig:ebeam_developed_100x} developed pattern at 100x magnification. Both developed using \acs{MIBK} and \acs{MEK}.}
	\label{fig:ebeam_developed}
\end{figure}

\subsection{Metal Deposition}\label{subsec:deposition}
In order to make proper electrical contacts metal must be deposited on the pattern made by \acs{EBL}. This process of metal deposition is done using a \ac{BJD} system which evaporates $10\unita{nm}$ of \ac{Ti} followed by $40\unita{nm}$ of \acs{Au} on the substrate`s surface at ultrahigh vacuum ($\sim 10^{-7}\unita{torr}$) at a rate of $\SI{1}{\angstrom}\unitb{s}{-1}$. After the metal deposition process is complete, a process called ``liftoff" is performed. In this process the substrate is placed in acetone for approximately 5-10 minutes, until the \acs{Au} has ``lifted-off." Once this process is complete all that remains on the substrate is \acs{Au} where the \acs{EBL} was performed. Figs.~\subref*{fig:liftoff_10x} and  \subref*{fig:liftoff_100x} illustrates a substrate that has undergone the liftoff process, all that remains is the portion of the pattern that is necessary to perform electrical measurements on the device. 
%\begin{figure}[ht]
%	\centering
%	\subfloat[BJD control panel]{
%		\includegraphics[height=6cm,width=4cm]{figs/experimental/bjd_control_panel}
%		\label{fig:bjd_control_panel}
%	}
%	\qquad
%	\subfloat[BJD Hood]{
%		\includegraphics[height=6cm,width=4cm]{figs/experimental/bjd_hood}
%		\label{fig:bjd_hood}
%	}
%	\caption[\acs{BJD} system]{\protect\subref{fig:bjd_control_panel} \acs{BJD} control panel, \protect\subref{fig:bjd_hood} \acs{BJD} hood.}
%	\label{fig:bjd_system}
%\end{figure}
\begin{figure}[ht]
	\centering
	\subfloat[]{
		\includegraphics[height=4cm,width=5cm]{figs/experimental/liftoff_10x}
		\label{fig:liftoff_10x}
	}
	\qquad
	\subfloat[]{
		\includegraphics[height=4cm,width=5cm]{figs/experimental/liftoff_100x}
		\label{fig:liftoff_100x}
	}
	\caption[\ch{Au}/\ch{Ti} deposited on device]{\protect\subref{fig:liftoff_10x} \ch{Au}/\ch{Ti} deposited on developed sample after electron beam lithography at 10x. \protect\subref{fig:liftoff_100x} \ch{Au}/\ch{Ti} deposited on developed sample after electron beam lithography at 100x.}
	\label{fig:liftoff}
\end{figure}


\section{Electrical Measurements and Characterization}\label{sec:measurements}
Upon completion of the fabrication process and physical characterization processes, the remaining process is electrical characterization. Many of the options available in electrical characterization are put to use, at least in some capacity regardless of the device. Then, depending on the device`s quality, further measurements can be made in which a more comprehensive electrical data profile can be obtained.
\subsection{Measurement Devices}\label{subsec:measurement_devices}
Pictured in fig.~\subref*{fig:measurement_setup} is a measurement device that allows for automated electrical data collection through pre-made programs. The user adjusts parameters and the desired device properties to be measured. Initially, the device is usually measured in non-vacuum conditions at room temperature in order to determine the device`s quality. If it is determined that the device`s quality is worth continuing with low temperature measurements then the substrate is placed in the vacuum measurement chamber pictured in fig.~\subref*{fig:vacuum_measurement}. This apparatus is used in conjuction with the measurement setup in fig.~\subref*{fig:measurement_setup} to continue measurements. However, in this configuration the measurements take place under ultrahigh vacuum ($\sim 10^{-6}$ to $10^{-7}\unita{torr}$) and temperature control via liquid nitrogen (with the ability to cool to $70\unita{K}$). In addition, there is another measurement device that allows for both temperature control and the application of a magnetic field. This device, known as a \ac{PPMS}, is shown in fig.~\subref*{fig:ppms} and is used in situations when it is necessary to apply a magnetic field such as measuring a Hall device.
\begin{figure}[ht]
	\centering
	\subfloat[]{
		\includegraphics[height=4cm,width=4cm]{figs/experimental/measurement_setup}
		\label{fig:measurement_setup}
	}
	\qquad
	\subfloat[]{
		\includegraphics[height=4cm,width=4cm]{figs/experimental/vacuum_measurement}
		\label{fig:vacuum_measurement}
	}
	\qquad
	\subfloat[]{
		\includegraphics[height=6cm,width=4cm]{figs/experimental/ppms}
		\label{fig:ppms}
	}
	\caption[Electrical measurement devices]{\protect\subref{fig:measurement_setup} Keithley semiconductor measurement system. \protect\subref{fig:vacuum_measurement} Low temperature, vacuum measurement chamber. \protect\subref{fig:ppms} Physical property measurement system.}
	\label{fig:measurement}
\end{figure}


