\begin{center}
\textbf{ABSTRACT}
	
	
	\singlespacing
\textbf{Intrinsic Channel Properties, Scattering Mechanisms, and Quantum Transport Properties in Transition Metal Dichalcogenides}\\
	\doublespacing
	
	by\\
	
	\textbf{Kraig J. Andrews}\\
	February 2016\\
\end{center}
\begin{tabular}{ll}	
Advisor: &Dr. Zhixain Zhou\\
Major:   &Physics\\
Degree:  &Doctor of Philosophy
\end{tabular}
\bigskip

\noindent Since 2004, the isolation of graphene by Novoselov \emph{et al.} the study of graphene and subsequent two-dimensional materials have garnered much interest. The discovery of graphene's lack of band gap led to researchers to search for other viable materials for digital applications. In light of this layered materials, especially transition metal dichalcogenides (TMDs), have been studied extensively. These TMDs are hexagonal structures layered with a metal sandwiched between a chalcogenide layer, for example \ch{MoS2} and \ch{WSe2}. These materials have interesting band structures with layer dependence and are considerably strong at the atomic level. With these facts in mind, TMDs have become one of the leading candidates for use in digital circuits, however, some critical challenges remain in order realize this goal.\\ 

\noindent In this study, we propose novel techniques to study the intrinsic properties of these TMDs, particularly \ch{WSe2} and \ch{MoS2}. In order to study these properties, we employ an approach to fabricating low-resistance contacts using degenerately $p$-doped \ch{WSe2}. Using this method, as shown in Chuang \emph{et al.}, resistances as low as $0.175\unita{k\Omega}\unita{\mu m}$. With the low-resistance contacts we study the intrinsic channel properties in $p$-doped \ch{WSe2} channel devices to understand and find how the doping content of the channel affects the device's performance for applications using \hbn enscapsulation. Using this configuration we find field-effect mobilities of $\sim 200\cmvs$ and $\sim 650\cmvs$ at $T=300\unita{K}$ and $T=5\unita{K}$, respectively. In addition, we study the intrinsic channel properties by manufacturing Hall bar devices. To this end, we hope to find the device performance limits for applications. By manufacturing high-quality devices we can study the quantum transport properties by measuring the device properties at low temperatures ($\sim 4\unita{K}$), namely the integer quantum Hall effect and the Shubnikov-de Haas oscillations. This will allow us to determine quantities like the quantum scattering times, effective cyclotron mass, and the geometry of the Fermi surface. 