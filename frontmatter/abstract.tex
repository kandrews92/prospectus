\begin{center}
\textbf{ABSTRACT}
	
	
	\singlespacing
\textbf{Quantum Transport Properties and Scattering Mechanisms in Transition Metal Dichalcogenides}\\
	\doublespacing
	
	by\\
	
	\textbf{Kraig J. Andrews}\\
	February 2016\\
\end{center}
\begin{tabular}{ll}	
Advisor: & Dr. Zhixain Zhou\\
Major:   &Physics\\
Degree:  &Doctor of Philosophy
\end{tabular}
\bigskip

\noindent Two-dimensional materials have garnered much interest since the isolation of graphene. Since then layered materials, such as transition metal dichalcogenides (TMDs) have been studied extensively. However, several key problems have yet to be solved. In this study we propose using methods developed to decrease and minimize contact resistance through a novel approach of 2D/2D contacts and \hbn encapsulation to study how the mobility and intrinsic properties are affect by $p$-doping \ch{WSe2} device channels. Preliminary results show contact resistance as low as $0.185\unita{k\Omega\cdot\mu m}$ using degenerately doped contacts and field-effect mobilities of $\sim 200\cmvs$ and $\sim 650\cmvs$ at $T=300\unita{K}$ and $T=5{K}$, respectively using this 2D/2D contact \hbn encapsulation method. In addition, this method is intended to improve mobility at both room temperature for device applications and at low temperatures ($\sim 4\unita{K}$) to study the integer quantum hall effect (IQHE) and the corresponding related quantum oscillations (Shubnikov-de Haas oscillations) to determine information related to quantum scattering times, effective cyclotron mass, and the geometry of the Fermi surface. 
