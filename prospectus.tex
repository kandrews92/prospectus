\documentclass[%
%   draft       % draft: filename in the header of every page, the date of
               %        printing follows the author's name, there is 
               %        no last page for the advisors signatures, and
               %        document spacing defaults to 2, both the abstract
               %        and the main body of the text.
   final       % final: the filename does not appear anywhere, same for the
               %        date of printing, the abstract (if exists) is in
               %        the single spacing, and the spacing of the main body
               %        of the text is determined by \LineSpacing, there is
               %        an additional (last) page with the places where the
               %        signatures of the advisory commettee go
]{prospectus}

\LineSpacing{1.5}% this can be a number between 1.5 and 2, to your choosing
                 % though it appears only in the final version, the draft
                 % is in double spacing

\begin{document}
\bibliographystyle{prsty}

\title{Theory of Atom-Molecule Bose Einstein Condensates}
\author{Marijan Ko\v{s}trun}
\MajorAdvisor{Juha Javanainen}     % these appear if the version is final
\AssociateAdvisorA{Robin C\^ot\'e} % on the last page with the place for
\AssociateAdvisorB{ ?/?}   % their signatures

% optional abstract of the prospectus, if not needed remove the whole
% environment
\abstract{%
  A quasi-continuum model of photo-association assumes an atomic Bose Einstein
  condensate (BEC) coherently coupled to a single mode (diatomic) molecular
  system in the presence of the laser field.
  In the ``mean-field'' formalism the Atom-Molecule Bose Einstein Condensate
  (AMBEC) is described by a system of two Gross-Pitaevskii equations,
  one for each BEC. Their interaction captures the coherent
  photo-associative/photo-dissociative  exchange of particles between the
  condensates: two atoms form a molecule, and a molecule splits into two
  atoms. We propose to analyze this system of equations numerically and
  theoretically. We expect our analytical
  research to reveal the general properties of the coupled AMBEC system,
  while the numerics should allow detailed analysis of any configuration of
  interest.
  It is our goal to find the extent to which the relevant physics is
  mirrored in our simple mathematical model, in particular, whether
  the model captures the experimentally observed collapse of a
  BEC as reported in J. Roberts et al., in Phys. Rev. Lett. 86, 4211
  (2001).
  }

\maketitle % creates a header of the document. but does not put it on a 
           % separate page

\section*{Background}

In 1924 and 1925, Bose and Einstein concluded that the ideal gas of
identical bosonic particles undergoes phase transition, 
if cooled to or below a critical temperature. 
At critical temperature the average 
particle's thermal (de Broglie) wavelength becomes comparable to the 
average interparticle spacing \cite{Bose1924,Einstein1925}. 
Further lowering of the temperature of the bosonic system causes particles to 
condense in the system's ground state, thus creating a macroscopic
quantum-mechanical object, a Bose-Einstein Condensate (BEC).

Seventy years later in 1995, the first experimental observations of
Bose Einstein condensation in dilute low-temperature atomic vapors were
reported \cite{Andersonetal1995,Davisetal1995}. 
These and subsequent experimental results have heightened the interest in the
theory of BEC.

\section*{Mean Field Theory of BEC}

The two pillars of the theory of BEC are the mean-field approximation by 
Bogoliubov \cite{Bogoliubov1947}, and the Gross-Pitaevskii equation.

The mean-field approximation is a theoretical approach present in many
branches of physics. It allows insight into the
behavior of the system in terms of parameters with clear physical meaning,
while avoiding heavy numerics associated with the more exact calculations

The mean-field theory of BEC was developed to tackle
a many-body problem of a large system of bosonic particles trapped in 
an external potential at the temperature of absolute zero. Such a system, is
ideally described by many-body Hamiltonian $\hat H$,
\begin{equation}
  \label{eq:nbodyproblem}
  \begin{array}[t]{rl}
  \hat H & = \int d {\bf r } \ \hat \Psi^{\dagger}( {\bf r} )
  \left( - \frac {\hbar^2} {2 \ m} \nabla^2 + V_{0}
    \right) \Psi ( {\bf r} ) \\
  & + \frac 1 2 \int d {\bf r }\ d {\bf r' } \ 
  \hat \Psi^{\dagger}( {\bf r} ) \ \hat \Psi^{\dagger}( {\bf r'} ) \ 
  V( {\bf r} - {\bf r'}) \ \hat \Psi( {\bf r'} ) \ \hat \Psi( {\bf r'} ),
  \end{array}
\end{equation}
where $\hat \Psi( {\bf r} )$ and $\hat \Psi^{\dagger} ( {\bf r} )$ are
the boson annihilation and creation field operators, and
$V( {\bf r} - {\bf r'})$ is a two-body interatomic
potential.  Equation (\ref{eq:nbodyproblem}), as is, is impossible to solve
analytically for any physically relevant number of particles ($10^6$ and 
higher). 

\noindent In Eq. (\ref{eq:nbodyproblem}) let us assume the following: the
bosonic system has undergone the Bose-Einstein condensation,
meaning that a quantum ground state contains the majority (if not all) of the 
system particles, and let the system be in thermodynamic limit, that is 
$N \rightarrow \infty$ and $V \rightarrow \infty$ but the limit
$\rho = \frac N V \rightarrow \rho_0$ exists, where $N$ is the number of 
particles and $V$ volume of the system.
These assumptions justify separation of the bosonic field operator
$\hat \Psi$ into two components: one pertaining to
the ground state, and the other describing the rest of the system.
The thermodynamic limit allows the operator corresponding to the ground state
to be replaced by a complex number. The general name of this complex number in
the language of the ``mean-field'' theories is ``order parameter'', and in
the case of BEC it is interpreted as ``density amplitude''.
The mean-field theory of BEC subsequently allows a development of the
``first-order'' theory for excitations in gaseous BEC,
\cite{Abrikosovetal1963}.

Consider further simplifications of the mean-field equations for the ground
state, that is, let us in Eq.(\ref{eq:nbodyproblem}) replace the
interaction potential 
$V({\bf r} - {\bf r'})$ 
by 
$U_0 \ \delta({\bf r} - {\bf r'})$. This yields a
 ``zeroth-order'' (or semiclassical) equation for the order parameter
$\Phi$, a BEC density amplitude, 
\begin{equation}
  \label{eq:gpe}
  i \hbar \frac {\partial \Phi( {\bf r}, t)}{\partial t} = \
  \left( - \frac {\hbar^2} {2 m} \nabla^2 + V_{0}({\bf r},t) \right)
  \Phi( {\bf r}, t) + 
  N U_0 \left| \Phi ({\bf r},t) \right|^2 \Phi( {\bf r}, t),
\end{equation}
where $m$ is the mass of a single particle while $N$ is the number of them.
This is known as the Gross-Pitaevskii (GP) equation
\cite{Gross1961,Pitaevskii1961,Gross1963}. It is valid
providing that the $s$-wave scattering length $a_0$, related to $U_0$ via
\begin{equation}
  \label{eq:1s}
  U_0 = \frac {4 \pi \hbar^2 a_0} {m}
\end{equation}
is much larger then the average distance between the atoms, and the
number of atoms in the condensate is
much larger than one \cite{Abrikosovetal1963,LifshitzPitaevskii1980}.
This condition is fulfilled in the dilute low-temperature atomic vapors.

If the atom-atom interaction in the above model is assumed repulsive,
$U_0 > 0$, the ground state and its properties can be calculated, 
(e.g., particle distribution and chemical potential),
see \cite{ParkinsandWalls1998} and the references therein.
In the opposite case, when the atom-atom interactions are attractive, 
$U_0 < 0$, 
it is known that for most of the confining geometries there exists a critical
number of particles that may be held in a metastable state. If the number
of particles is larger than the critical value, a collapse of the BEC
distribution to the trap center is predicted.

\section*{Recent Results}


Recent experiments \cite{Robertsetal2001,Robertsetal2001B,Cornishetal2000}
have achieved a high accuracy in tuning of the $s$-wave scattering
length in trapped
$^{85}$Rb BECs. In all the experiments the control over the scattering length
was obtained using Feshbach resonances, \cite{Feshbach1950,Child1974},
in an external magnetic field. In
particular, Ref. \cite{Robertsetal2001B} reports on the existence of
$(i)$ a critical magnetic field at which a $s$-wave scattering length 
reaches a critical negative value, and $(ii)$ a phenomenon the authors termed
``collapse'' that occurs upon decreasing the magnetic field below the 
threshold. As a result of the collapse the number of atoms in BEC decreases
substantially, while the ratio of axial to radial width varies erratically
from the initial ratio.
The authors
concluded that as a result of the collapse, the condensate oscillates in a
highly excited state, while loosing an undetermined number of the most
energetic particles.

Apparently, the properties of the collapse obtained by solving a simple
Gross-Pitaevskii equation differ from those observed
experimentally. This inadequacy of GPE is emphasized further by the
results of Ref. \cite{glc}. There it was shown that a quantum 
many-body system of particles described
by GPE (\ref{eq:gpe}) with the attractive atom-atom interaction, $U_0 < 0$,
undergoes a gas-liquid phase transition. This intuitively plausible result
claims, in plain words, that the collapse of a BEC, according to GPE,
produces a droplet of liquid.

\section*{Proposed Research}

All experimental realizations of BEC in atomic vapors were done with
atoms of group $IA$ in the periodic table.  A pair of these atoms may form a
molecule, and the theory of formation of such diatomic molecules has been
extensively studied. 
More recently, a production of the molecules in ultracold atomic collisions 
became the focus of theoretical research, 
\cite{Tiesingaetal1993,Moerdijketal1995}. 
Typically, such a molecule creation process is analyzed in
terms of Feshbach resonances, i.e., the bound motion in one
degree of freedom (molecule) is coupled weakly to and can decay into unbound
motion in another degree of freedom (pair of atoms) 
\cite{friedrich94:feshbach}. 
The mechanism of Feshbach resonances has been suggested to be applicable
for the case of atomic BEC in an external magnetic field, where it produces
molecules which are both
Bose condensed and in coherence with the atomic BEC. This conjecture allows
the GPE to be written for the combined system of atom-molecule BEC
(AMBEC) \cite{Timmermansetal1999}.

As an alternative to Feshbach resonances, the quasi-continuum
photo-association based on quantum field theory 
(QCPA) was suggested, \cite{jjonpa1998}. In this approach,
one considers a quasi-continuum (QC) of free states coupled to a single bound
state in the presence of a detuned laser field. 
In Feshbach resonances 
molecules are Bose condensed in the trap's ground state, while in QCPA
the phase of the  molecules is modulated by a factor $e^{i \, {\bf q \, x}}$,
as a result of the recoil of PA photon with momentum {\bf q}. 
The phase modulation is not all that important, 
because the nature of molecular BEC does not change, i.e., a single
quantum state (with momentum ${\bf q}$) is
macroscopically occupied.
Finally, the QCPA, unlike Feshbach resonances, allows easier identification of
the relevant physical parameters in a simple model of AMBEC:
$K$, the atom-molecule coupling strength and
$\delta$, the laser energy mismatch from resonance.

While both approaches result in a formally identical coupled two-mode system
of GP equations, we choose the formalism and terminology of
QCPA for the reasons mentioned above. This said, the system of equations 
describing a simple model of AMBEC is:
\begin{equation}
  \label{eq:ambec}
  \begin{array}[t]{rcl}
  i \hbar \frac {\partial \varphi}{\partial t} &=&
  H_{0,a} \varphi - K \varphi^* \psi, \\
  i \hbar \frac {\partial \psi}{\partial t} &= &
  \left( H_{0,m}-\delta \right)  \psi - 
  \frac K 2 \varphi^2,
  \end{array}
\end{equation}
where $H_{0,a}$ and $H_{0,m}$ are the Hamiltonians for atomic and molecular
harmonic-oscillator like trapping potentials, respectively. The important
aspect of  Eq. (\ref{eq:gpe}) is that the adiabatic expansion
with respect to detuning $\delta$ allows elimination of the molecular field
$\psi$, yielding the GPE for atomic BEC only,
\begin{equation}
  \label{eq:adiabatic}
  i \hbar \frac {\partial \varphi}{\partial t} \simeq
  H_{0,a} \varphi - \frac {K^2} {\delta}  |\varphi|^2 \varphi.
\end{equation}
This expression indicates the possibility of changing the effective
$s$-wave scattering length of the BEC atoms by changing the
detuning $\delta$ and the atom-molecule coupling strength $K$. Even the sign
change of the $s$-wave scattering length is possible
by simply changing the sign of detuning. 
Naturally, the question arises if and how Eq. (\ref{eq:ambec}) relates to
the problem of collapse: is it in accord with experimentally 
observed phenomena of Refs.
\cite{Robertsetal2001,Robertsetal2001B,Cornishetal2000}, or perhaps follows
the collapse path of the single GPE (\ref{eq:gpe}) discussed earlier?

We propose to examine these questions analytically and numerically.

For the numerical part of the problem,  we have developed a novel
time-integration method for the nonlinear Hamiltonians \cite{Kostrun2001:DS}. 
The method, called the {\it DS-method},
can be modified to calculate the ground state, by integrating in 
complex instead of the real time. Our intention
to explore the $\{ \delta, K \}$ parameter space of
the Eq.(\ref{eq:ambec}) is motivated by our preliminary results for
the one-dimensional problem, where we found a puzzling lack of any collapse
whatsoever.
We would like to determine whether the same holds for
two- and three-dimensional systems, and for different initial configurations
of the AMBEC. 
In parallel, we intend to extend our numerical work, written mainly in
combination of RLaB and C/C++, as to allow the detuning $\delta$ to be time
dependent and for PA matrix element $K$ to depend both on time and the
position.
The goal we have in mind is to determine whether these results confirm
the predictions stemming from adiabatic expansion (\ref{eq:adiabatic}) 
or not.

While numerical results should provide the details of time evolution
of any particular AMBEC configuration, we surmise that certain general laws
can be deduced about the system. We distinguish between the of free and
trapped AMBEC.

For the trapped system, our main concern is the structural stability of
the system (\ref{eq:ambec}), i.e., whether it collapses or not. 
In order to perform this analysis, we will first determine
the equations of motion of the functional $\eta$ given by
\begin{equation}
  \label{eq:eta}
  \eta(\tau) = 
  \left< \varphi | {\bf x}^2 | \varphi \right> +
  \left< \psi    | {\bf x}^2 | \psi  \right>,
\end{equation}
from (\ref{eq:ambec}). This procedure will result in 
differential equations from which general results might be
obtained. 
Our approach is an extension of the work on the collapse of a single component
BEC, \cite{Wadatietal1998}, that utilizes the Zakharov's collapse analysis
method, Refs. \cite{Zakharov1972,Zakharovetal1975}.
In the formalism of this method, a collapse happens when
$\eta \rightarrow 0$.

In the case of free condensates, we intend to extend the known results
from the GP theory of the single component BEC \cite{Drummondetal1998}.
Primarily, a repulsive atom-atom interaction makes a condensate structurally
unstable and prone to modulation instability, while conversely, an
attractive atom-atom interaction is instrumental in forming stable
soliton-shaped distributions.

\section*{Outlook}

Upon successful completion of the work described here, we
should be able to answer the following questions: First, to what extent is it
possible to manipulate the $s$-wave scattering length of the particles in a
BEC? Second, to what extent can one control the behavior of the condensates 
by manipulating the $s$-wave scattering length?

% references in separate file(s) all with extension .bib if using bibtex
% or with preamble \bibitem if using plain latex
\begin{thebibliography}{10}

\bibitem{Bose1924}
S. Bose, Zeit. Phys. {\bf 26},  178  (1924).

\bibitem{Einstein1925}
A. Einstein, Sitzber. Kgl. Preuss. Akad. Wiss. {\bf 3},    (1925).

\bibitem{Andersonetal1995}
M. Anderson {\it et~al.}, Science {\bf 269},  198  (1995).

\bibitem{Davisetal1995}
K. Davis {\it et~al.}, Phys. Rev. Lett. {\bf 75},  3969  (1995).

\bibitem{Bogoliubov1947}
N. Bogoliubov, J. Phys. (Moscow) {\bf 11},  23  (1947).

\bibitem{Abrikosovetal1963}
A. Abrikosov, L. Gorkov, and I. Dzyaloshinski, {\em Methods of Quantum Field
  Theory in Statistical Physics} (Prentice-Hall, Inc., Englewood Cliffs, New
  Jersey, 1963).

\bibitem{Gross1961}
E. Gross, Nuovo Cimento {\bf 20},  454  (1961).

\bibitem{Pitaevskii1961}
L. Pitaevskii, Zh. Eksp. Teor. Fiz. {\bf 40},  646  (1961), english translation
  in Sov. Phys. JETP {\bf 13}, 451 (1961).

\bibitem{Gross1963}
E. Gross, J. Math. Phys. {\bf 4},  195  (1963).

\bibitem{LifshitzPitaevskii1980}
E. Lifshitz and L. Pitaevskii, {\em Statistical Physics, Part II} (Pergamon
  Press, Oxford, 1980).

\bibitem{ParkinsandWalls1998}
A. Parkins and D. Walls, Phys. Rep. {\bf 303},  1  (1998).

\bibitem{Robertsetal2001}
J. Roberts {\it et~al.}, Phys. Rev. Lett. {\bf 81},  5109  (1998), arXiv:
  physics/0104053.

\bibitem{Robertsetal2001B}
J. Roberts {\it et~al.}, Phys. Rev. Lett. {\bf 86},  4211  (2001), arXiv:
  cond-matt/0102116.

\bibitem{Cornishetal2000}
S. Cornish {\it et~al.}, Stable $^{85}$Rb Bose-Einstein Condensates with Widely
  Tunable Interactions, arXiv: cond-matt/0004290.

\bibitem{Feshbach1950}
H. Feshbach, Ann. Phys. {\bf 5},  357  (1958).

\bibitem{Child1974}
M. Child, {\em Molecular Collision Theory} (Dover Publications, Inc., Mineola,
  N.Y., 1996), work first published by Academic Press, London 1974, Vol. 4 of
  the monograph series "Theoretical Chemistry".

\bibitem{glc}
S. ichiro Koh, Attractive Boson and the Gas-Liquid Condensation, arXiv:
  cond-mat/0009471.

\bibitem{Tiesingaetal1993}
E. Tiesinga, B. Verhaar, and H. Stoff, Phys. Rev. A {\bf 47},  411  (1993).

\bibitem{Moerdijketal1995}
A. Moerdijk, B. Verhaar, and A. Axelsson, Phys. Rev. A {\bf 51},  4852  (1995).

\bibitem{friedrich94:feshbach}
H. Friedrich,  in {\em Theoretical Atomic Physics}, 2nd ed. (Springer, Berlin,
  1994), Chap.~1, pp.\ 30--52.

\bibitem{Timmermansetal1999}
E. Timmermans, P. Tommasini, M. Hussein, and A. Kerman, Phys. Rep. {\bf 315},
  199  (1999).

\bibitem{jjonpa1998}
J. Javanainen and M. Mackie, Phys. Rev. A {\bf 58},  R789  (1998).

\bibitem{Kostrun2001:DS}
M. Ko\v{s}trun and J. Javanainen, J. Comp. Phys. {\bf 172},  298  (2001).

\bibitem{Wadatietal1998}
M. Wadati and T. Tsurumi, Phys. Lett. A {\bf 247},  287  (1998).

\bibitem{Zakharov1972}
V. Zakharov, Sov. Phys. JETP {\bf 35},  908  (1972).

\bibitem{Zakharovetal1975}
V. Zakharov and V. Synakh, Sov. Phys. JETP {\bf 41},  465  (1975).

\bibitem{Drummondetal1998}
P. Drummond, K. Kheruntsyan, and H. He, Phys. Rev. Lett. {\bf 81},  3055
  (1998).

\end{thebibliography}

\backmatter % if the option is final, then this produces an extra page where
            % the approval statement with the advisors signatures go

\end{document}






