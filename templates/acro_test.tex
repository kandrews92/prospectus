\documentclass{article}

\usepackage[printonlyused]{acronym}

\title{Example of the Acronym package in use}
\author{Hugh C. Pumphrey}

\begin{document}
\maketitle
\section{Description}
The acronym package is useful for handling such things as \ac{MLS} and
\ac{HiRDLS}. The first time you mention an acronym, it gets
expanded. Subsequent uses do not get expanded, so if I mention
\ac{MLS} again, then only the acronym appears. The package does not
detect your acronyms magically: it is
up to the user to write \ac{MLS} as \verb+\ac{MLS}+ every time it
appears. The package warns you if you use an acronym that is not
defined. For example, if I do \verb+\ac{OMI}+  I  get \ac{OMI}, and a
warning 
\begin{verbatim}
Package acronym Warning: Acronym `OMI' is not defined on input line 17.
\end{verbatim}
is printed to the terminal.

The acronyms themselves are defined within the \texttt{acronym}
environment. If you load the package with \verb+\usepackage{acronym}+
then all of the acronyms you define will appear in the table of
acronyms, whether or not you use them. If, on the other hand, you load
the package with \verb+\usepackage[printonlyused]{acronym}+ then the
printed table only consists of acronyms that you actually used
somewhere. 

\acresetall
Finally, you can use  \verb+\acresetall+ if you want the definitions
of acronyms to appear again. There is an  \verb+\acresetall+ just
above this sentence, so if I use \verb+\ac{MLS}+ here I get this: \ac{MLS}.

\section*{List of Acronyms}
\begin{acronym}
\acro{MLS}{Microwave Limb Sounder}
\acro{HiRDLS}{High-Resolution Dynamics Limb Sounder}
\acro{TES}{Tropospheric Emission Sounder}
\end{acronym}


\end{document}