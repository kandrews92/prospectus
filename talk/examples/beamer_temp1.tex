% $Header: /cvsroot/latex-beamer/latex-beamer/solutions/generic-talks/generic-ornate-15min-45min.en.tex,v 1.4 2004/10/07 20:53:08 tantau Exp $

\documentclass{beamer}

% This file is a solution template for:

% - Giving a talk on some subject.
% - The talk is between 15min and 45min long.
% - Style is ornate.


\mode<presentation>
{
  \usetheme{Warsaw}
  % or ...

  \setbeamercovered{transparent}
  % or whatever (possibly just delete it)
}


\usepackage[english]{babel}
% or whatever

\usepackage[latin1]{inputenc}
% or whatever

\usepackage{times}
\usepackage[T1]{fontenc}
% Or whatever. Note that the encoding and the font should match. If T1
% does not look nice, try deleting the line with the fontenc.


\title[Lecture 8] % (optional, use only with long paper titles)
{Mathematics 1000, Winter 2008}

\subtitle
{Lecture 8} % (optional)

\author[S Zhang] % (optional, use only with lots of authors)
{Sheng Zhang} %\inst{1}} % \and S.~Another\inst{2}}
% - Use the \inst{?} command only if the authors have different
%   affiliation.

%%%  \institute[Wayne State University] % (optional, but mostly needed)
%%%  {
  %%%  %\inst{1}%
  %%%  Department of Mathematics\\
  %%%  Wayne State University}
  %%%  % \and
  %%%  % \inst{2}%
  %%%  % Department of Theoretical Philosophy\\
  %%%  % University of Elsewhere}
%%%  % - Use the \inst command only if there are several affiliations.
%%%  % - Keep it simple, no one is interested in your street address.

\date[Short Occasion] % (optional)
{February 4, 2007}

\subject{MAT1000}
\begin{document}

\begin{frame}
  \titlepage
\end{frame}

\begin{frame}
  \frametitle{Today's Topics}
  \tableofcontents
\end{frame}

\section{Review of correlation}

\begin{frame}{Properties of $r$}
\begin{itemize}
\item
Always between $-1$ and $1$
\vfill
\item
$r > 0 $  $\longrightarrow$  positive relation
\item
$r < 0 $  $\longrightarrow$  negative relation
\vfill
\smallskip
\item
Close to $1$ or close to $-1$ $\longrightarrow$  strong correlation
\item
Close to $0$ $\longrightarrow$ weak correlation
\end{itemize}
\end{frame}

\begin{frame}{More properties of $r$}
\begin{itemize}
\item
Does not depend on units used
\pause
\vfill
\item Can be affected by outliers
\end{itemize}
\end{frame}

\begin{frame}{Caution}
Think about whether the value of
$r$ calculated by your calculator makes sense in view of
the picture as well as the context.
\end{frame}

\section{Regression lines}

\begin{frame}{Warning}

The book writes the equation of the regression line as
                                  $$y = a + bx$$

Many calculators use    $y = ax + b$

\vfill
\alert{These are inconsistent!}  The meanings of $a$ and $b$
are reversed.

\vfill
If you are used to    $y = mx + b$  then your $b$ is
the same as the calculator's.  It is not the book's $b$.  
\end{frame}

\begin{frame}{The Least-Squares Regression Line}

This is the line that `best' fits the data in a specific,
technical sense.  

\vfill

The slope of this line is positive when the the correlation is
positive. 

\vfill
But the size of the slope depends on the scale used for the
parameters.

\vfill
The size of the correlation $r$ tells you how well the straight
regression line fits the data.
\end{frame}

\begin{frame}{Example:  Beer and Blood Alcohol Concentration}
$r = .89$,  $m = .018$
\includegraphics[width=3.5in]{beers.pdf}

\end{frame}

%%% 
%%% \begin{frame}{Large correlation, small slope}
%%% The correlation $r$ is $0.8943$.
%%% 
%%% The slope $m$ of the regression line is
%%% $$\displaystyle r \frac{s_y}{s_x} = 0.8943 \times \frac{0.04414}{2.1975}$$
%%% 
%%% This is approximately $0.01796$.
%%% 
%%% In this example, the correlation is large (.89).
%%% 
%%% \vfill
%%% 
%%% But the slope of the regression line is only .018.
%%% \end{frame}

%%%  \begin{frame}{Formula for least squares regression line} 
%%%  $$y = mx + b$$
%%%  where
%%%  $$m = r \frac{s_y}{s_x}$$
%%%  and
%%%  $$b = \bar{y} - m \bar{x}$$
%%%  %%%  %%%  
%%%  \vfill
%%%  $r =$ correlation \\
%%%  $s_x =$  standard deviation for $x$ \\
%%%  $s_y =$  standard deviation for $y$
%%%  
%%%  \vfill
%%%  $\bar{x} = $ average $x$ value \\
%%%  $\bar{y} = $ average $y$ value
%%%  \end{frame}

\begin{frame}{Relation between slope and correlation}
$$m = r \frac{s_y}{s_x}$$

\noindent
$m =$ slope \\
$r =$ correlation\\
$s_x =$  standard deviation for $x$ \\
$s_y =$  standard deviation for $y$

\vfill
If $s_y$ is much larger than $s_x$, then $m$ can be big
even if $r$ is small.

\vfill
If $s_y$ is much smaller than $s_x$, then $m$ can be small
even if $r$ is large.
\end{frame}

\begin{frame}{Remember}

There will be such a least squares regression line, whether or not it has any significance!

\vfill
When it is meaningful, we can use the regression line to
make predictions.

\vfill
Interpretation is critical.
\end{frame}

\begin{frame}{Example}

\includegraphics[width=4in]{speedfuel2.pdf}
%  << Excel chart >>  "Speed and fuel consumption" with trendline
%  correlation = -0.17

\end{frame}

\section{Interpretation} 
\subsection{Is there really a correlation?}
\begin{frame}{Interpreting Correlation and Regression}


The regression line can be strongly influenced by
a small number of outliers, sometimes even by just
one outlier.

\vfill
To minimize the sum of squares, when there is 
a clump of points and one outlier, the trend line
will join that outlier to the middle of the clump.  
\end{frame}


\begin{frame}{Outliers Example 1}
\includegraphics[width=4in]{exp1.pdf}
\end{frame}

\begin{frame}{Outliers Example 1, with regression line}
\includegraphics[width=4in]{exp1L.pdf}
\end{frame}

\begin{frame}{Outliers Example 2}
\includegraphics[width=4in]{exp2.pdf}
\end{frame}

\begin{frame}{Outliers Example 2, with regression line}
\includegraphics[width=4in]{exp2L.pdf}
\end{frame}

\begin{frame}{Outliers Example 3}
\includegraphics[width=4in]{exp3.pdf}
\end{frame}

\begin{frame}{Outliers Example 3, with regression line}
\includegraphics[width=4in]{exp3L.pdf}
\end{frame}

\begin{frame}{Marriage-Divorce Rates, All States}
\includegraphics[width=4in]{divorceall.pdf}
\end{frame}
%  \end{document}

\begin{frame}{Marriage-Divorce Rates, w/o Hawaii and Nevada}
\includegraphics[width=4in]{divorcenovegas.pdf}
\end{frame}

\subsection{Causal relations}

\begin{frame}{Causal relations}

We have a scatterplot and a correlation.
What can we conclude from this?

\vfill
If the correlation is close to 0 or is strongly
influenced by outliers, then the correct answer is
``not much.''

\vfill
We'll focus now on situations where the correlation
is high and is not influenced by outliers.
\end{frame}

\begin{frame}{Be careful!}
This stage can be especially tricky and contentious.

\vfill
People seem to have a tendency to see a cause when
there may be only a high correlation.
\end{frame}

\begin{frame}{Example:  Pocket change}
There is a statistical correlation between
an individual's height and the amount of money in
that person's pocket.

\pause
\vfill
Why do taller people tend to carry more money in
their pockets?

\pause
\vfill
Are they less afraid of being robbed?

\pause
\vfill
Or is there some other factor at work?
\end{frame}

\begin{frame}{The underlying cause}
There are two types of people.

\pause
\vfill
One type tends to be taller and carry money in their
pockets.

\pause
\vfill
The other type tends to be shorter and carry money in
their purses.
\end{frame}


\begin{frame}{Some slightly  more subtle examples}

From the book:

\begin{itemize}
\item
Children who listen to Mozart tend to score higher
in tests of verbal skills.

\item
Countries where people spend more time watching
television have higher life expectancies.

\item
People who drink large amounts of diet soda
are heavier for their height than people who 
do not.
\end{itemize}
\end{frame}

\begin{frame}{Example:  a cancer scare}

A 1979 study found a strong correlation between 
exposure to strong electromagnetic fields and 
incidence of childhood leukemia.

\vfill
This was initially interpreted as evidence that
these fields posed a health hazard.

\vfill
More extensive investigations have indicated
that ``although some of the correlations remain
we [researchers at the University of Texas] now 
do not think that this type of radiation causes 
cancer.''

\vfill
{\tt \normalsize
% http://
www.utexas.edu/courses/bio301d/Topics/EMF/Text.html
}
\end{frame}

\begin{frame}{A tough call}
Determining the direction of the causal relationship
is sometimes tricky.

\vfill
Health and wealth are correlated, but which causes which?
\end{frame}

\begin{frame}{Global warming}
Is there a link between human activities, such as burning
fossil fuels (e.g., gasoline), and increasing global temperatures?

\vfill
This has been a highly contentious question.

\vfill
Last week, the IPCC, the Intergovernmental Panel on Climate
Change, issued a report that concluded there is almost certainly
such a causal link.

\end{frame}

\begin{frame}
\includegraphics[width = 4in]{ipcc22.pdf}
\end{frame}

\begin{frame}
For most people, this should be a definitive answer to the
question.

\vfill
It is not difficult to find people with apparently good credentials
who dispute that conclusion.
\end{frame}

\begin{frame}
\includegraphics[width=4in]{marshall.pdf}
\end{frame}


\begin{frame}
That the oil industry lobbying group
is offering generous payments to people who will
make such statements is of course rather suspicious.  
\end{frame}

\begin{frame}
\includegraphics[width=4in]{bribe2.pdf}
\end{frame}

\begin{frame}
Unfortunately, sorting through sophisticated, but possibly
dishonest, claims is beyond the current scope of this course.

\vfill
The best we can hope for this semester is to be able to spot some
of the cruder fakes.
\end{frame}

\begin{frame}{Looking ahead}

Chapter 7 concerns how to gather data.

\vfill
If you want to know something about all of the
states, you can just list them all.

\vfill
But if you want to know something about all of
the students at Wayne State University, it is 
probably impractical to ask every one of them.
\end{frame}

\begin{frame}{Sampling techniques}

We'll be looking at sampling techniques, ways
to look at a relatively small number of cases in
order to get information about the whole.

\vfill
We'll see where the `margin of error' comes from
in survey results and polls.
\end{frame}

\end{document}


**********************

Slide 1:
Lecture 8

Slide 2:
Announcements
The Chapter 6 Quiz will be Wednesday, October 4.
Chapter 7 homework assignments are now posted on the website.
The four week deadline is Monday.  After October 2, you cannot get a W grade in any class at Wayne State University.


Slide 3:
The Least-Squares Regression Line

This is the line that “best” fits the data in a specific,
technical sense.  

The equation of the line can be calculated using the
correlation, the standard deviations, and the means
of the variables.

Calculators and computers have built in programs to
do this calculation.

Slide 4:

<< pdf file >>   "The trendline minimizes the sum of the squares ..."

Slide 5:

The least-squares regression line always exists.

Sometimes, it is not very useful.

Here’s an example from an exercise (#7) that was
not part of the homework.

Slide 6:

<< Excel chart >>  "Speed and fuel consumption"

Slide 7:

<< Excel chart >>  "Speed and fuel consumption" with trendline
correlation = -0.17

Slide 8:

If we use the regression line to calculate the fuel
consumption at 160 km/hr, which is about 100 mph,
we get

y = -0.0147 x 160  + 11.058
   = 8.71 liters per 100 km

This is about 2.3 gallons per 100 miles,
more than 40 miles per gallon.


Slide 9:

Based on the graph, we would not expect this
to be a realistic estimate.

However, if we restrict our attention to speeds
of 60 km/hr (roughly 40 mph) or more, the
data points do align much better.

Slide 10:

<< Excel chart >>  "Speed and fuel consumption, highway speeds"

Slide 11:

<< Excel chart >>  "Speed and fuel consumption, highway speeds"
  with trendline

correlation  =  0.99

Slide 12:

Note the improvement in correlation, from -.17 to .99.

Working with this set of data, we can make a prediction
for the gas mileage at 100 mph with considerably more
confidence.


Slide 13:

If we use the new regression line to calculate the fuel
consumption at 160 km/hr, which is about 100 mph,
we get

y = -0.0773 x 160  + 0.8131
   = 13.18 liters per 100 km

This is about 3.5 gallons per 60 miles,
about 17 miles per gallon.


Slide 14:

Interpreting Correlation and Regression

Rule number one:  Watch out for outliers!

The regression line can be strongly influenced by
a small number of outliers, sometimes even by just
one outlier.

Slide 15:

<< Excel chart >>  "Number of Beers Consumed and BAC"
Actual data


Slide 16:
<< Excel chart >>  "Number of Beers Consumed and BAC"
(different from above)
One incorrect data point

Slide 17:

To minimize the sum of squares, when there is 
a clump of points and one outlier, the trend line
will join that outlier to the middle of the clump.

Slide 18:

<< Excel chart >>  unlabeled, 7 points 0 - 250 by 0 - 120

Slide 19:

<< Excel chart >>  unlabeled, 7 points 0 - 250 by 0 - 120
with trendline

correlation = .98


Slide 20:

<< Excel chart >>  unlabeled, different 7 points 0 - 250 by 0 - 120

Slide 21:

<< Excel chart >>  as in Slide 20,
with trendline
correlation = -.61

Slide 22:

<< Excel chart >>  unlabeled, a third set of 7 points 0 - 250 by 0 - 120

Slide 23:

<< Excel chart >>  as in Slide 22,
with trendline
correlation = .97

Slide 24:

Interpretation

We have seen that strong correlations can be influenced
by outliers.

Another thing to watch for is imputing a causal 
relationship to what appears to be a genuinely high
correlation.

Slide 25:

One famous statistical correlation is the one between
an individual’s height and the amount of money in
that person’s pocket.

Why do taller people tend to carry more money in
their pockets?

Are they less afraid of being robbed?

Or is there some other factor at work?

Is there a type of person who tends to be taller
and also carry more pocket change?


Slide 26:

In this case, it isn’t hard to figure out what is going
on, but sometimes it’s harder to do so.

The book discusses several famous examples:

Children who listen to Mozart tend to score higher
in tests of verbal skills.

Countries where people spend more time watching
television have higher life expectancies.

People who drink large amounts of diet soda
are heavier for their height than people who 
do not.

Slide 27:

A 1979 study found a strong correlation between 
exposure to strong electromagnetic fields and 
incidence of childhood leukemia.

This was initially interpreted as evidence that
these fields posed a health hazard.

More extensive investigations have indicated
that “although some of the correlations remain
we [researchers at the University of Texas] now 
do not think that this type of radiation causes 
cancer.”

http://www.utexas.edu/courses/bio301d/Topics/EMF/Text.html

Slide 28:

Determining the direction of the causal relationship
is sometimes tricky.

For example, a study on the health and wealth of
nations considered the positive correlation between 
health and per capita income (which has long been 
known), raising the possibility that the wealth was
a consequence of the health, contrary to the 
conventional wisdom that wealth determined health.

Slide 29:

Looking ahead

Chapter 7 concerns how to gather data.

If you want to know something about all of the
states, you can just list them all.

But if you want to know something about all of
the students at Wayne State University, it is 
probably impractical to ask every one of them.

Slide 30:

We’ll be looking at sampling techniques, ways
to look at a relatively small number of cases in
order to get information about the whole.

We’ll see where the “margin of error” comes from
in survey results and polls.

